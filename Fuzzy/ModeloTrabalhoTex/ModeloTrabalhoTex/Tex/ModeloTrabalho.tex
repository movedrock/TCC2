\documentclass[english]{article}
\usepackage[T1]{fontenc}
\usepackage[latin1]{inputenc}

\makeatletter \providecommand{\tabularnewline}{\\}

\usepackage{cec2003,multicol,times}
\usepackage{epsfig}
\usepackage{graphicx}
\usepackage{hyphenat}
\usepackage{indentfirst}
\usepackage{babel}
\usepackage{eufrak}
\usepackage{psfrag}
\makeatother
\begin{document}

\pagestyle{empty} \sloppy \twocolumn[
\title{T�TULO DO TRABALHO}
\begin{multicols}{3}
\begin{center}
\textbf{Autor 1} \\
UTFPR\\
Av. Sete de Setembro, 3165 \\
80230-901 Curitiba, Brazil \\
email \\\vspace{2cm}
\end{center}
\begin{center}
\textbf{Autor 2} \\
 UTFPR\\
Av. Sete de Setembro, 3165 \\
80230-901 Curitiba, Brazil \\
email \\\vspace{2cm}
\end{center}
\begin{center}
\textbf{...Autor N} \\
UTFPR\\
Av. Sete de Setembro, 3165 \\
80230-901 Curitiba, Brazil \\
email \\\vspace{2cm}
\end{center}
\end{multicols}
]


\begin{abstract}
O trabalho dever� ser em duas colunas de acordo com o modelo aqui
apresentado. Dever� ter entre 4 e 6 p�ginas contendo as se��es
e/ou subse��es listadas a seguir. \vspace{2cm}
\end{abstract}

\section{Introdu��o}
\vspace{1cm}
 Breve introdu��o posicionando  o tema da pesquisa.
Identifique o contexto geral no qual o problema ou as quest�es da
pesquisa foram identificadas. \vspace{2cm}


\subsection{Descri��o do Problema}
\vspace{1cm} O que ser� pesquisado? O que se vai fazer?
\vspace{2cm}
\subsection{Motiva��o}
\vspace{1cm} Por que se deseja fazer a pesquisa? \vspace{1cm}

Identifique os argumentos que justifiquem que a sua pesquisa �
significativa, importante e/ou relevante. \vspace{2cm}
\section{Objetivos}
\vspace{0.5cm} Para que se deseja fazer a pesquisa? \vspace{1cm}

Identifique claramente o que se pretende alcan�ar com a pesquisa.
\vspace{2cm}
\section{Revis�o da Literatura}
\vspace{1cm} Mostrar o enfoque recebido pelo tema j� publicado na
literatura. \vspace{1cm}

An�lise comentada do que j� foi escrito sobre o tema procurando
mostrar os enfoques convergentes e divergentes dos diversos
autores. \vspace{1cm}

 Exemplo de cita��o das refer�ncias
bibliogr�ficas\cite{zadeh65,zimmermann80,takagi85,takagi93,klement00,klir}
\vspace{2cm}
\section{Metodologia}
\vspace{1cm} Como sera realizada a pesquisa? \vspace{1cm}

Descrever detalhadamente a abordagem proposta. \vspace{2cm}
\section{Simula��es e Resultados}
\vspace{1cm} � interessante que a abordagem possa ser simulada e
os resultados sejam apresentados nesta se��o. Neste caso, o modelo
simulado baseado em fuzzy deve ser avaliado e, se for o caso,
comparado com outro modelo, para verificar se h� vantagem no uso
de sistemas fuzzy. \vspace{2cm}
\section{Conclus�es}
\vspace{1cm} Esta se��o dever� trazer as conclus�es a respeito da
abordagem e resultados obtidos. \vspace{2cm}

\bibliographystyle{WCCI}
\bibliography{reffuzzy}

\end{document}
